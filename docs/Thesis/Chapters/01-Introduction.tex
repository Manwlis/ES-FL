\chapter{Introduction}
\label{Chapter-Introduction}

In recent years, edge devices with advanced computing and data collection capabilities are becoming commonplace. As a result, massive volumes of new and useful data are generated, which can be exploited in Machine Learning (ML). When combined with recent advances and techniques in ML, new opportunities emerge in a variety of fields, including self-driving automobiles and medical applications. 

Traditional ML approaches demand the data to be consolidated in a single entity where learning takes place. However, due to unacceptable latency and storage requirements of centralizing huge amounts of raw data, this may be undesirable. To address the inefficiency of data silos, cloud computing architectures such as Multi-access edge computing (MEC) \cite{MEC} have been proposed in order to transfer the learning closer to where the data is produced. Unfortunately, these techniques still require raw data to be shared between the edge devices and intermediate servers.

Due to growing privacy concerns, recent legislation like General Data Protection Regulation (GDPR) \cite{GDPR} and California Consumer Privacy Act (CCPA) \cite{CCPA} have severely limited the usage of technologies that transfer private data. To continue leveraging the increasing real-world data while adhering to such regulations, the concept of Federated Learning (FL) \cite{FL-original-paper} has been introduced. FL is a collaboratively decentralized privacy-preserving technology, in which learning takes place at the data collection point, i.e. the edge device. The edge devices train a ML model provided by the server and share model updates instead of raw data. As a result, collaborative and distributed ML is possible while maintaining the privacy of the participating devices.

\section{Motivation}

Most FL research, to our knowledge, focuses on simulations and treats edge devices as black boxes; generally ignoring their nature and constrains. Taking in consideration the complexities of implementing ML on hardware, recent advancements in FL might be diminished or invalidated. The main motivation of this thesis is to identify, explore and possibly overcome the intrinsic conflicts that exist between FL and Artificial Neural Network (ANN) training in Field Programmable Gate Arrays (FPGA)s. % Such conflicts can be the batch size, where in FL tends to be minimized.

Instead of being incompatible, these two technologies may complement each other, which is worth investigating. Frequently in FL, transformations are applied on the generated ANN variables to reduce network utilization and enhance privacy. These transformations, which include quantization \cite{Mills2020}, adding Gaussian noise \cite{Wei2020} and others, tend to be spatially independent and could be implemented highly efficiently in hardware accelerators like FPGAs.

Finally, FL literature is almost devoid of wall-clock time examples. This thesis aims to provide a real world FL implementation that may be considered as a benchmark for future research. Furthermore, in order to be extendable and utilized in future works, the FL implementation is modular and platform independent.


\section{Scientific Contributions}

The main focus of this thesis is combining FL training with FPGA based ANN implementations, while exploring and overcoming their inherent conflicts. Furthermore, it focus on the mostly unexplored FL setting of small client pools and its implicit difficulties. Finally, it provides a real world implementation of FL that can be used as a benchmark for future works. This FL implementation is agnostic of the ANN training implementation and can be used as a starting point for future works.

\section{Thesis Outline}
% Todo: Fill chapter descriptions
\begin{itemize}
    \item \textbf{Chapter 2 - Theoretical Background:} Description of the theoretical background of ML and FL.
    \item \textbf{Chapter 3 - Related Work:} Related works on FL, optimization techniques and hardware implementations of it.
    \item \textbf{Chapter 4 - FL architecture \& design:} Description of the FL architecture, design and implementation developed.
    \item \textbf{Chapter 5 - Robustness Analysis:} Analysis of the quality and performance of the FL implementation.
    \item \textbf{Chapter 6 - FPGA Implementation:} Description of the ANN architecture, design and implementation on FPGA developed.
    \item \textbf{Chapter 7 - Results:} Analysis of the quality and performance of the complete system.
    \item \textbf{Chapter 8 - Conclusions and Related Work:} Chapter 8 description % TODO: this
\end{itemize}
